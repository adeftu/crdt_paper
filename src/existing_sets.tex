\begin{algorithm}[t]
\small{
	\floatname{algorithm}{Specification}
	\caption{G-Set (state-based)}
 	\label{alg:g_set_state_based}                       

 	\begin{algorithmic}[1]
 	  \State \Payload $A = \varnothing$
 	  
 	  \State \Update $add(e)$
 	  \State \hspace{\algorithmicindent} $A \coloneqq A \cup \{e\}$
 	  
 	  \State \Query $lookup(e) : \text{boolean}$
 	  \State \hspace{\algorithmicindent} \Return $e \in A$
 	  
 	  \State \Compare $(S, T) : \text{boolean}$
 	  \State \hspace{\algorithmicindent} \Return $S.A \subseteq T.A$
 	  
 	  \State \Merge $(S, T) : \text{payload}$
 	  \State \hspace{\algorithmicindent} \Return $S.A \cup T.A$
	\end{algorithmic}
 }
\end{algorithm}
\begin{algorithm}[t]
\small{
	\floatname{algorithm}{Specification}
	\caption{2P-Set (state-based)}
 	\label{alg:2p_set_state_based}                       

 	\begin{algorithmic}[1]
 	  \State \Payload $A = \varnothing, R = \varnothing$
 	  
 	  \State \Update $add(e)$
 	  \State \hspace{\algorithmicindent} $A \coloneqq A \cup \{e\}$
 	  
 	  \State \Update $remove(e)$
 	  \State \hspace{\algorithmicindent} \Pre $lookup(e)$
 	  \State \hspace{\algorithmicindent} $R \coloneqq R \cup \{e\}$
 	  
 	  \State \Query $lookup(e) : \text{boolean}$
 	  \State \hspace{\algorithmicindent} \Return $e \in A \land e \not\in R$
 	  
 	  \State \Compare $(S, T) : \text{boolean}$
 	  \State \hspace{\algorithmicindent} \Return $S.A \subseteq T.A \lor S.R \subseteq T.R$ 
 	  
 	  \State \Merge $(S, T) : \text{payload}$
 	  \State \hspace{\algorithmicindent} \Let $U.A = S.A \cup T.A$
 	  \State \hspace{\algorithmicindent} \Let $U.R = S.R \cup T.R$
 	  \State \hspace{\algorithmicindent} \Return $U$
	\end{algorithmic}
 }
\end{algorithm}

\section{Existing Replicated Set Designs}
\label{sec:existing_sets}

The set data type is a pervasive abstraction, used either directly or as a
building block in more complex types, such as maps or graphs. The supported
update operations are \textit{add} and \textit{remove} which add and,
respectively, remove an element to and from the set. These operations do not
commute and, therefore, a CRDT set will be only an approximation to the
sequential specification of a set. Many designs for replicated sets have been
proposed~\cite{shapiro:inria-00555588} and they mainly differ in which operation
takes precedence in a $\textit{add}(e) \parallel \textit{remove}(e)$ situation.
In this section we give examples of some specifications and argue that the
OR-Set does not present any anomaly as the others do, which could lead to a
counter-intuitive behavior.

\begin{algorithm}[t]
\small{
	\floatname{algorithm}{Specification}
	\caption{OR-Set (op-based)}
 	\label{alg:or_set_op_based}                       

 	\begin{algorithmic}[1]
 	  \State \Payload $S = \varnothing$
 	  \State \Query $lookup(e) : \text{boolean}$
 	  \State \hspace{\algorithmicindent} \Return $\exists u : (e, u) \in S$
 	  
 	  \State \Update $add(e)$
 	  \State \hspace{\algorithmicindent} \Prepare $(e) : \text{tag}$
 	  \State \hspace{\algorithmicindent}\hspace{\algorithmicindent} \Let $\alpha = unique()$
 	  \State \hspace{\algorithmicindent}\hspace{\algorithmicindent} \Return $\alpha$
 	  \State \hspace{\algorithmicindent} \Effect $(e, \alpha)$ 
 	  \State \hspace{\algorithmicindent}\hspace{\algorithmicindent} $S \coloneqq S \cup \{(e, \alpha)\}$
 	  
 	  \State \Update $remove(e)$
 	  \State \hspace{\algorithmicindent} \Prepare $(e) : \text{set}$
 	  \State \hspace{\algorithmicindent}\hspace{\algorithmicindent} \Pre $lookup(e)$
 	  \State \hspace{\algorithmicindent}\hspace{\algorithmicindent} \Let $R = \{(e, u) \mid \exists u : (e, u) \in S\}$
 	  \State \hspace{\algorithmicindent}\hspace{\algorithmicindent} \Return $R$
 	  \State \hspace{\algorithmicindent} \Effect $(R)$ 
 	  \State \hspace{\algorithmicindent}\hspace{\algorithmicindent} \Pre $\forall (e, u) \in R : add(e, u)$ has been delivered
 	  \State \hspace{\algorithmicindent}\hspace{\algorithmicindent} $S \coloneqq S \setminus R$
	\end{algorithmic}
 }
\end{algorithm}
\begin{algorithm}[t]
\small{
	\floatname{algorithm}{Specification}
	\caption{OR-Set (state-based)}
 	\label{alg:or_set_state_based}                       

 	\begin{algorithmic}[1]
 	  \State \Payload $A = \varnothing, R = \varnothing$
 	  \State \Query $lookup(e) : \text{boolean}$
 	  \State \hspace{\algorithmicindent} \Return $\exists \alpha : (e, \alpha) \in A \land \nexists \beta : (a, \alpha, \beta) \in R$
 	  
 	  \State \Update $add(e)$
 	  \State \hspace{\algorithmicindent} \Let $\alpha = unique()$
 	  \State \hspace{\algorithmicindent} $A \coloneqq A \cup \{(e, \alpha)\}$
 	  
 	  \State \Update $remove(e)$
 	  \State \hspace{\algorithmicindent} \Pre $lookup(e)$
 	  \State \hspace{\algorithmicindent} \Let $\beta = unique()$
 	  \State \hspace{\algorithmicindent} $R \coloneqq R \cup \{(e, \alpha, \beta) | \exists (e, \alpha) \in A\}$
 	  
 	  \State \Compare $(S, T) : \text{boolean}$
 	  \State \hspace{\algorithmicindent} \Return $S.A \subseteq T.A \land S.R \subseteq T.R$ 
 	  
 	  \State \Merge $(S, T) : \text{payload}$
 	  \State \hspace{\algorithmicindent} \Let $U.A = S.A \cup T.A$
 	  \State \hspace{\algorithmicindent} \Let $U.R = S.R \cup T.R$
 	  \State \hspace{\algorithmicindent} \Return $U$
	\end{algorithmic}
 }
\end{algorithm}

\textbf{G-Set}. The first CRDT implementation of a set, \textit{Grow-Only Set},
allows for \textit{add} and \textit{lookup} operations. Both state-based and
op-based versions have a set as payload. To prove that this is a CmRDT it is
easy to see that $\textit{add}(e)$ is commutative, being based on a set-union
operation between the payload and $\{e\}$. In the state-based approach, as shown
in Specification~\ref{alg:g_set_state_based}, the partial order on states $S$
and $T$ is given by $S \sqsubseteq T \iff S \subseteq T$. Then, the
\textit{merge} operation defined as $\textit{merge}(S,T) = S \cup T$ computes
the LUB in the monotonic semilattice $(S, \sqsubseteq)$. And so, G-Set is also a
CvRDT\footnote{For proofs on the rest of the constructs, the reader is
referred to \cite{shapiro:inria-00555588}.}.

\textbf{2P-Set}. A \textit{Two-Phase Set} brings the option to remove an
element. However, once an element has been removed, it cannot be added again to
the set. The principle is to use two G-Sets, one for adding and another for
removing (also known as \textit{tombstone set}). Removing an element is
conditioned by being present in the set at source. The state-based variant is
shown in Specification~\ref{alg:2p_set_state_based}. The payload consists of set
$A$ for adding and set $R$ for removing. Adding or removing the same element
twice or adding an already removed element has no effect.

\textbf{U-Set}. If we guarantee that each element in the set is unique and that
an $\textit{add}(e)$ is delivered before $\textit{remove}(e)$, the tombstone set
becomes redundant and can be discarded because the causal delivery criteria
is met. This new data structure is called \textit{U-Set}.

\textbf{PN-Set}. An alternative solution is to associate with each element a
CRDT counter (initially set to 0) which is increased when the element is added
to the set and decreased when it is removed. If the counter is negative, it
means that the element is not in the set, and \textit{add} operation will not
have any effect. Thus a PN-Set has the anomaly that after adding a previously
removed element to an empty set, it remains empty. This may not always be the
intended semantics, despite the fact that PN-Set converges: it combines two
CRDTs, a set and a counter.

\textbf{OR-Set}. The previously described set structures, although practical,
have counter-intuitive behaviors. For example, the 2P-Set does not allow adding
an element after it has been removed, while the PN-Set has the problem showed
above. The \textit{Observed-Removed Set}, introduced in
\cite{shapiro:inria-00555588}, is closer to the usual set semantics. The new
approach is to uniquely tag each added element. When removing an element, only
associated tags observed at the source are removed.

Specification~\ref{alg:or_set_op_based} describes the usual supported operations
for the op-based variant. The payload is a set of pairs
$(\textit{e},\textit{tag})$. Method $\textit{add}(e)$ generates a new unique tag
at source in the prepare-update phase and then sends it to all replicas which
insert it into their payload in the effect-update phase. In this way, two
additions of the same element are distinguished by their tags, but
\textit{lookup} masks the duplicates. Method $\textit{remove}(e)$ gathers all
tags associated with $e$ at source and sends them to all replicas which remove
the corresponding pairs from their local payloads. Because a
$\textit{remove}(e)$ will only remove locally observed elements, a concurrent
$\textit{add}(e) \parallel \textit{remove}(e)$ will give precedence to
$\textit{add}(e)$, in contrast to the 2P-Set.

The state-based approach is presented in
Specification~\ref{alg:or_set_state_based}. Here the payload contains two sets,
$A$ for added elements and $R$ for removed elements. When adding an element $e$,
like in the op-based approach, a new unique tag $\alpha$ is generated and the
pair $(e, \alpha)$ is inserted into the $A$ set. The \textit{remove} operation
again generates a unique tag $\beta$, associates it with all matching pairs from
$A$, and stores the result in the $R$ set. To test if an element is in the
OR-Set, we just need to verify if it is in $A$ and not in $R$.
