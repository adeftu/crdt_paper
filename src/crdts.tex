\section{Conflict-free Replicated Data Types}
\label{sec:crdts}

% - consistency models (strong, eventual, strong eventual)
% - CRDTs model: 2 approaches, comparison
% - examples of CRDTs

Achieving consistency is one of the hardest problems in the field of distributed
systems. What we ideally want to have is replicas which are \textit{strongly
consistent}, meaning that any update happening at one replica is made
instantaneously visible at all the others. Because this approach implies
synchronization after each update operation, essentially leading to a serial
execution, it is rarely used in practice. \textit{Eventual
consistency}~\cite{DBLP:journals/queue/Vogels08a,Saito:2005:OR:1057977.1057980}
is a weaker form that moves the synchronization phase out of the critical path,
to the background. In this way, updates can always be made locally, even if the
network is partitioned. However, it still requires conflict arbitration
techniques, such as a consensus algorithm or
roll-backs~\cite{Terry:1995:MUC:224056.224070}.

\textit{Conflict-free Replicated Data Types} (CRDTs) introduce the concept of
\textit{Strong Eventual Consistency}
(SEC)~\cite{Shapiro:2011:CRD:2050613.2050642}.
