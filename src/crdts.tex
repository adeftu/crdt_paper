\section{Conflict-free Replicated Data Types}
\label{sec:crdts}

Achieving consistency is one of the hardest problems in distributed
systems. What we ideally want to have is replicas which are \textit{strongly
consistent}, meaning that any update happening at one replica is made
instantaneously visible at all the others. Because this approach implies
synchronization after each update operation, essentially leading to a serial
execution, it is rarely used in practice. \textit{Eventual
consistency}~\cite{DBLP:journals/queue/Vogels08a,Saito:2005:OR:1057977.1057980}
is a weaker form that moves the synchronization phase out of the critical path,
to the background. In this way, updates can always be made locally, even if the
network is partitioned. However, it still requires conflict arbitration
techniques, such as a consensus algorithm or
roll-backs~\cite{Terry:1995:MUC:224056.224070}.

\textit{Conflict-free Replicated Data Types} (CRDTs) introduce the concept of
\textit{Strong Eventual Consistency}
(SEC)~\cite{Shapiro:2011:CRD:2050613.2050642}. The idea is to design the data
structures such that all updates have deterministic outcome, and thus they can 
be made immediately persistent. In this way, conflicts are altogether avoided.
There are two styles for defining CRDTs. 

\textbf{State-based replication}. In this approach, each update operation is
executed entirely at the source replica, modifying its state. Subsequently,
every replica occasionally sends its local state to some other replica, which
\textit{merges} it into its own state. Convergence is achieved by eventually
delivering every update directly or indirectly to all replicas. There are well
known protocols in the literature that do this in a fault-tolerant manner, such
as gossip or anti-entropy~\cite{Demers:1987:EAR:41840.41841,
Petersen:1997:FUP:268998.266711}.

State-based style uses the concept of \textit{semilattice} defined as follows. A
\textit{partially ordered set} is the pair consisting in a set together with a
binary relation $\sqsubseteq$ which establishes an order between the elements of
the set. A \textit{least upper bound} (LUB) $\sqcup$ is defined as follows: for
any elements $x$, $y$ from the set, $m = x \sqcup y$ is LUB of $\{x, y\}$ if and
only if $x \sqsubseteq m$ and $y \sqsubseteq m$ and there is no other $m'
\sqsubseteq m$ such that $x \sqsubseteq m'$ and $y \sqsubseteq m'$. From this
definition, it follows that a LUB is: i) commutative: $x \sqcup y = y \sqcup x$,
ii) idempotent: $x \sqcup x = x$, iii) associative: $(x \sqcup y) \sqcup z = x
\sqcup (y \sqcup z)$. A partially ordered set which has a LUB is called a
\textit{join-semilattice}~\cite{semilattice} (or just \textit{semilattice} from
now on). An example of semilattice is $(2^{\{x, y, z\}}, \subseteq)$, where
$2^{\{x, y, z\}}$ is the power set of $\{x, y, z\}$ and the LUB is given by
set-union.

In order to be SEC, a state-based object needs to fullfill the following
conditions:
\begin{itemize}
  \item Its payload takes values in a semilattice $(S,\sqsubseteq)$.
  \item Updates monotonically increase the states according to $\sqsubseteq$.
  \item The \textit{merge} operation computes the LUB between the local and
  remote states.
\end{itemize}

An example of state-based object can be one whose payload is an integer value,
$\sqsubseteq$ is common the integer order $\leq$, and where $merge()
\stackrel{def}{=} max()$.

\textbf{Operation-based (op-based) replication}. Here, the system transmits only
update operations across replicas, and not whole states. Applying an update is
split in two phases. The first one has no side-effects and its role is to
compute some intermediary results. When it terminates, the system sends the
update operation, its parameters, and possible intermediary results from the
first phase to all replicas, including the source. Here, in the second phase,
called \textit{downstream}, the operation is executed immediately at source and
asynchronously at all other replicas if a downstream precondition is
met\footnote{An example of such precondition could be that an element is allowed
to be removed from a replicated set only if it is present in the set at source.}.
This second phase cannot return results and has to execute atomically.

An update $u$ is said to \textit{happen-before} an update $u'$ at some replica,
denoted by $u \rightarrow u'$, if $u$ has been applied when $u'$ executes. Two
updates are \textit{concurrent} if no one happens before the other: $u \parallel
u' \iff u \not\rightarrow u' \land u' \not\rightarrow u$. The conditions for an
op-based object to achieve SEC are:
\begin{itemize}
  \item Related updates by happened-before, $u \rightarrow u'$, are applied in
  the same order at all replicas: first $u$ and then $u'$.
  \item Concurrent operations, $u \parallel u'$, \textit{commute}: executing $u$
  immediately followed by $u'$ leads to the same state as executing $u'$
  immediately followed by $u$. Concurrent operations may be delivered in any
  order.
\end{itemize}

There are common epidemic protocols, such as Bayou's
anti-entropy~\cite{Petersen:1997:FUP:268998.266711}, which guarantee this
causal ordering for updates delivery.

Interestingly, these two replication styles are
equivalent~\cite{Shapiro:2011:CRD:2050613.2050642}: any data type that can be
implemented as a state-based object can also be implemented as an op-based
object and vice versa. However, there are trade-offs which should be considered
for each approach. State-based objects have the advantage of being simpler to
reason about, given that all the information is captured by their state.
Moreover, because this state is transmitted as part of the \textit{merge}
procedure between two replicas, updates may be lost along the way, applied
multiple times, or in different orders. This also constitutes a disadvantage
since sending whole states is an expensive network operation for large
objects. On the other hand, op-based objects are more complex to specify since
we need to reason about history and to deliver the updates in a
causal-consistent manner according to \textit{happens-before}. This puts
pressure on the communication channel which has to be reliable and has to
guarantee the same order of message delivery to all replicas. The advantage
comes in the form of a greater expressive power for type specification, e.g.
there is no \textit{merge} method which has to compute a LUB, and smaller
payloads.

Examples of current CRDT implementations~\cite{shapiro:inria-00555588} include:
a) replicated counters: G-Counter, PN-Counter, Non-negative, b) registers, or
mutable shared variables: LWW-Register, MV-Register, c) sets: G-Set, 2P-Set,
U-set, PN-Set, OR-Set, d) graphs: 2P2P-Graph, Add-only monotonic DAG. 
