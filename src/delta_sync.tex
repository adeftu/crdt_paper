\section{A Delta-based Synchronization Algorithm}
\label{sec:delta_sync}

As seen in Section~\ref{sec:existing_sets}, there are different ways of
constructing a CRDT set data structure.  OR-Sets are very intuitive and do not
suffer from the semantics anomalies encountered in the other set specifications.
Our goal is therefore to have the robustness of state-based OR-Set corroborated
with the transfer efficiency of the op-based one. This section introduces our
first contribution: improvement to the state-based OR-Set specification to
transfer only deltas between replicas instead of full states, called
\textit{delta-based synchronization} (\textit{merging})
algorithm\footnote{Synchronization here has the meaning of updates propagation
between replicas, and not that of a consensus required in the case of strong
consistency model.}, in Specification~\ref{alg:or_set_with_deltas}.

\begin{algorithm}[t]
\small{
	\floatname{algorithm}{Specification}
	\caption{OR-Set with delta-based synchronization}
 	\label{alg:or_set_with_deltas}                       

 	\begin{algorithmic}[1]
 	  \State \Payload $A = \varnothing, R = \varnothing, T = [\,]$
 	  
 	  \State \Query $lookup(e) : \text{boolean}$
 	  \State \hspace{\algorithmicindent} \Return $\exists (e, t, r) \in A \land \nexists (e, t, r, t', r') \in R$
 	  
 	  \State \Update $add(e)$
 	  \State \hspace{\algorithmicindent} \Let $r = replica()$
 	  \State \hspace{\algorithmicindent} \Let $t = T[r] + 1$
 	  \State \hspace{\algorithmicindent} $A \coloneqq A \cup \{(e, t, r)\}$
 	  \State \hspace{\algorithmicindent} $T[r] \coloneqq  t$

 	  \State \Update $remove(e)$
 	  \State \hspace{\algorithmicindent} \Pre $lookup(e)$
 	  \State \hspace{\algorithmicindent} \Let $r' = replica()$
 	  \State \hspace{\algorithmicindent} \Let $t' = T[r'] + 1$
 	  \State \hspace{\algorithmicindent} $R \coloneqq R \cup \{(e, t, r, t', r') \mid \exists (e, t, r) \in A\}$
 	  \State \hspace{\algorithmicindent} $T[r'] \coloneqq  t'$

 	  \State \Compare $(S_{1}, S_{2}) : \text{boolean}$
 	  \State \hspace{\algorithmicindent} \Return $S_{1}.A \subseteq S_{2}.A \land S_{1}.R \subseteq S_{2}.R \land S_{1}.T[i] \leq S_{2}.T[i], \forall i$

 	  \State \Merge $(S_{1}, S_{2}) : \text{payload}$
 	  \State \hspace{\algorithmicindent} \Let $A' = \{(e, t, r) \in S_{2}.A \mid S_{1}.T[r] < t\}$
 	  \State \hspace{\algorithmicindent} \Let $R' = \{(e, t, r, t', r') \in S_{2}.R \mid S_{1}.T[r'] < t'\}$
 	  \State \hspace{\algorithmicindent} \Let $P.A = S_{1}.A \cup A'$
 	  \State \hspace{\algorithmicindent} \Let $P.R = S_{1}.R \cup R'$
 	  \State \hspace{\algorithmicindent} \Let $P.T = max(S_{1}.T, S_{2}.T)$
 	  \State \hspace{\algorithmicindent} \Return $P$
	\end{algorithmic}
 }
\end{algorithm}

In addition to the original OR-Set, the payload now has also a \textit{timestamp
vector} $T$ which has as many components as there are replicas and for which
$T[r]$ records the latest known version of replica $r$. For this purpose, it is
assumed that each replica has a unique identifier that can be retrieved through
the function \textit{replica} and that $T$ can be indexed with this identifier.
Adding a new element $e$ at replica $r$ increments the corresponding component
$T[r]$ to obtain $t$ and inserts the tuple $(e, t, r)$ into set $A$. Compared to
the basic OR-Set, the change was essentially to split the tag which uniquely
identified each element into the pair $(t, r)$. In this way, the elements still
remain tagged, but now we also have the information about the partial order of
updates occurring at each replica, i.e. we know that tuple $(e, t, r)$ was added
before tuple $(e', t', r)$ at replica $r$ if $t < t'$. Removing an element uses
the same principle. Being an OR-Set data type, only locally observed elements at
the source are removed. The logical clock corresponding to the replica is
increased again to keep track of this update. Looking up an element $e$ in the
set translates to verifying if there is an added tuple containing $e$ and does
not exist a corresponding remove tuple. In order to merge, we first send $T$ to
the other side, compute here the missing updates in $A'$ and $R'$, i.e. tuples
whose timestamps are greater than the logical clock, send them back together
with the remote $T$, and finally append the updates and update the local
timestamps. 

Therefore, $T$ acts as a version vector~\cite{Parker:1983:DMI:1313337.1313753},
which guarantees the partial order between updates. Also, due to the
transitivity property of the version vectors, each $\textit{merge}(S_{1},
S_{2})$ includes not only the updates originated at $S_{2}$ but also those from
$S_{3}$ which were pulled by $S_{2}$ but not by $S_{1}$. One limitation of this
approach is that indexing $T$ requires a static mapping from a global replica
identifier to an integer. Dynamic version vector maintenance using interval tree
clocks~\cite{Almeida:2008:ITC:1496310.1496330} may alleviate this problem
however.

\begin{IEEEproof}[Delta-based synchronization maintains the CRDT
properties] 
We consider the partial order $(S, \sqsubseteq)$, where $\sqsubseteq$ is given
by the \textit{compare} method in the specification. Both \textit{add} and
\textit{remove} methods add elements to the payload and increment $T$ and
therefore advance the state in the partial order. Furthermore, for any two sets
$X$ and $Y$, it is known that the following holds: $X \cup Y = X \cup (Y
\setminus X)$. So \textit{merge} basically computes the union of the added and,
respectively, removed sets using the right-hand side formula and the maximum of
the two timestamp vectors. Hence we have $\textit{merge}(S_{1}, S_{2}) = S_{1}
\sqcup S_{2}$ (LUB).
\end{IEEEproof}