\section{Comparison with Previous Work}
\label{sec:previous_work}

The fundamental principles on database replication are laid out
in~\cite{lindsay} and a number of techniques are discussed there to achieve
consistency. The traditional \textit{strong consistency} approach imposes a
global total order on updates to serialize
them~\cite{Lamport:1978:TCO:359545.359563}. This conflicts with availability and
partition-tolerance~\cite{Gilbert:2002:BCF:564585.564601} and leads to
performance and scalability bottlenecks. \textit{Sequential consistency} is
another model, weaker than strong consistency, but undecidable in
practice~\cite{Qadeer:2003:VSC:939835.940001}. A survey on other models is
presented in~\cite{Mosberger:1993:MCM:160551.160553}.

Techniques for achieving eventual consistency for large-scale distributed
systems have been an active focus point in recent research. This is mostly due
to the explosion of Internet-based and peer-to-peer services. However, the
origins of the principles behind CRDTs can be found in the apparent unrelated
area of file systems. The state-based approach was introduced for register-like
objects, where the only operation is assignment. It is widely used in
NFS~\cite{Sandberg85designand} and AFS~\cite{Howard:1988:SPD:35037.35059} file
systems and in key-value stores such as Amazon's
Dynamo~\cite{DeCandia:2007:DAH:1294261.1294281}. The mathematical foundations
were laid by Baquero and Moura~\cite{scadt4} and later extended by Shapiro and
Pregui\c{c}a in their work on Treedoc~\cite{Preguica:2009:CRD:1584339.1584604}
in order to support the operation-based approach, thus coining the term of CRDT.
Examples of implementations for this second approach are found in Bayou's
anti-entropy protocol~\cite{Petersen:1997:FUP:268998.266711} and the IceCube
cooperative system~\cite{preguica:inria-00445758}.

Later, a formal definition and rigorous system model for CRDT were published
in \cite{shapiro:inria-00555588} and \cite{Shapiro:2011:CRD:2050613.2050642}.
These are the first works to engage a comprehensive and systematic study on
CRDTs. 