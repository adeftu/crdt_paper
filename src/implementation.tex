\section{Library Implementation}
\label{sec:library_implementation}

To test these concepts, we implemented a Java client library for the SOR-Set
which can connect to any replica cluster and provide access to the usual set
operations: adding, removing, looking up an element, or synchronizing with other
replica clusters.

For the database server we used Redis~\cite{redis}, a widely used, open-source,
in-memory, key-value store. Redis data model is a dictionary that maps keys to
values. The keys can be only strings, while values can be strings, lists, sets,
sorted sets, or hashes. Important Redis features include persistence,
replication, transactions, pipelining, and Lua scripting. In addition, it has
support for associating timeouts with keys: after the timeout has expired, the
key is automatically deleted. Our client communicates with Redis via the
Jedis library~\cite{jedis}.

The payload for each shard is stored in a separate Redis database, or
\textit{store}, using the schema from
Listing~\ref{lst:redis_database_schema_for_sor-set}, where underlined words
represent hard-coded strings and non-underlined ones are to be replaced with
their corresponding values.

\begin{algorithm}[t]
\setcounter{algorithm}{0}
\small{
	\floatname{algorithm}{Listing}
	\caption{Redis database schema for a SOR-Set shard}
 	\label{lst:redis_database_schema_for_sor-set}                       
 	\begin{algorithmic}[1]
 	  \State \underline{timestamp}:$rc$:$rs$ $\rightarrow$ $t$ \Comment{Integer string}
 	  \State \underline{element}:$rc.rs.id$ $\begin{aligned}[t]
                                              & \rightarrow \text{\underline{value}} \rightarrow e \\[-2pt]
                                              & \rightarrow \text{\underline{add.t}} \rightarrow t \\[-2pt]
                                              & \rightarrow \text{\underline{add.rc}} \rightarrow rc \\[-2pt]
                                              & \rightarrow \text{\underline{add.rs}} \rightarrow rs \\[-2pt]
                                              & \rightarrow \text{\underline{rmv.t}} \rightarrow t' \\[-2pt]
                                              & \rightarrow \text{\underline{rmv.rc}} \rightarrow rc' \\[-2pt]
                                              & \rightarrow \text{\underline{rmv.rs}} \rightarrow rs'
                                            \end{aligned}$ \Comment{Hash}
 	  \State \underline{index}:$rc$:$rs$ $\rightarrow$ [$t$:$rc.rs.id$] \Comment{List}
 	  \State \underline{ids}:$e$ $\rightarrow$ {$rc.rs.id$} \Comment{Set}
 	  \State \underline{element:next.id} $\rightarrow$ $id$ \Comment{Integer string}
	\end{algorithmic}
 }
\end{algorithm}

Each cell of the vector of timestamps, $T[rc][rs]$, is stored at key
\underline{timestamp}:$rc$:$rs$. Instead of using different sets for added and
removed tuples, we combine and store them together as \textit{elements}.
An element contains: i) the string value $e$, ii) information about a adding:
timestamp $t$ and ids for the source replica cluster and shard, $(rc, rs)$, iii)
similar information for removing: $t', rc', rs'$. Each element is stored in the
hash at key \underline{element}:$rc.rs.id$, where $rc.rs.id$ represents a global
unique id: $rc$ and $rs$ are the ids which uniquely identifies the source shard
and $id$ is a per-shard counter stored at \underline{element:next.id} key which
is incremented with each new element insertion.

In Specification~\ref{alg:sor_set} of the SOR-Set, the \textit{merge} method
filters all tuples added or removed after a given timestamp. For this purpose,
an index is kept as a list of element ids sorted by their timestamp. With each
add or remove of an element at shard $(rc, rs)$, its id is appended to the list
\underline{index}:$rc$:$rs$. Since the index is kept per shard and timestamps at
each shard are monotonically increasing (adding and removing always increases
the local timestamp), \underline{index}:$rc$:$rs$ is guaranteed to be always
sorted. Thus, filtering new elements is very efficient. Adding the same
value $e$ multiple times to the set creates a new element for each operation. A
second index stored at \underline{ids}:$e$ keeps all the element ids
corresponding to value $e$, needed for $\textit{remove}(e)$ and
$\textit{lookup}(e)$ methods.

\textbf{Add}. Listing~\ref{alg:add} gives the pseudocode for \textit{add}
operation. First, the logical clock of the shard and the local id counter are
incremented. \texttt{incr} is an internal Redis command which increments the
number stored at the specified key by one\footnote{The rest of Redis commands will not be
described as their usage will be easily deduced from the context.}. Next, a
hash is created to store the new element and its expiration time is set. Last
two lines update the two indices previously discussed.

\begin{algorithm}[t]
    \small{
    \floatname{algorithm}{Listing}
    \caption{Redis SOR-Set: \textit{add}}
    \label{alg:add}
    \centering
    \begin{algorithmic}[1]
 	  \Procedure{Add}{$e$, $rc$, $rs$, $ttl$}
 	    \State $t$ $\gets$ \texttt{incr} \underline{timestamp}:$rc$:$rs$
 	    \State $id$ $\gets$ \texttt{incr} \underline{element:next.id}
 	    \State \texttt{hmset} \underline{element}:$rc.rs.id$ $\begin{aligned}[t]
                                                               & \text{\underline{value} } e \\[-2pt]
                                                               & \text{\underline{add.t} } t \\[-2pt]
                                                               & \text{\underline{add.rc} } rc \\[-2pt]
                                                               & \text{\underline{add.rs} } rs
                                                              \end{aligned}$
        \State \texttt{expire} \underline{element}:$rc.rs.id$ $ttl$
        \State \texttt{lpush} \underline{index}:$rc$:$rs$ $t$:$rc.rs.id$
        \State \texttt{sadd} \underline{ids}:$e$ $rc.rs.id$
 	  \EndProcedure
	\end{algorithmic}
    }
\end{algorithm}

\textbf{Remove}. The \textit{remove} method is described in
Listing~\ref{alg:remove}. Again, removing an element $e$ from an SOR-Set
consists in getting all $ADD(e)$ tuples and tagging them as removed. Thus, on
line~3 all element ids for value $e$ are retrieved using index
\underline{ids}:$e$. For each element stored at \underline{element}:$gid$,
if it is not yet expired, i.e. the key still exists in the database, the
corresponding fields are populated. Finally, the procedure updates the
expiration period of the element and pushes the new timestamp together with the
id to the index list. After removal, the new timestamp $t'$ will be ahead of the
old one, $t$, in this list, which the expected behavior: the remove happened
after the add.

\begin{algorithm}[t]
    \small{
    \floatname{algorithm}{Listing}
    \caption{Redis SOR-Set: \textit{remove}}
    \label{alg:remove}
    \centering
    \begin{algorithmic}[1]
 	  \Procedure{Remove}{$e$, $rc'$, $rs'$, $ttl$}
 	    \State $t'$ $\gets$ \texttt{incr} \underline{timestamp}:$rc'$:$rs'$
 	    \State $ids$ $\gets$ \texttt{smembers} \underline{ids}:$e$
 	    \ForAll{$gid$ \In $ids$}
 	      \If{\texttt{exists} \underline{element}:$gid$}
 	        \State \texttt{hmset} \underline{element}:$gid$ $\begin{aligned}[t]
                                                              & \text{\underline{rmv.t} } t' \\[-2pt]
                                                              & \text{\underline{rmv.rc} } rc' \\[-2pt]
                                                              & \text{\underline{rmv.rs} } rs'
                                                              \end{aligned}$
            \State \texttt{expire} \underline{element}:$gid$ $ttl$
            \State \texttt{lpush} \underline{index}:$rc'$:$rs'$ $t'$:$gid$
 	      \EndIf
 	    \EndFor
 	  \EndProcedure
	\end{algorithmic}
    }
\end{algorithm}

\textbf{Lookup}. The \textit{lookup} method searches for the existence of at
least one $ADD(e)$ tuple for which there is no corresponding $RMV(e)$. If there
is such tuple, then element $e$ is in the set. This is exactly what
Listing~\ref{alg:lookup} does. First, ids for all elements $e$ are retrieved.
Next, for each element $(e, t_1, rc_1, rs_1, t_1', rc_1', rs_1')$, if it is an
$ADD(e)$ tuple ($t_{1}' = null$, its removed timestamp is not set) and does not
exist any corresponding $RMV(e)$ tuple $(e, t_2, rc_2, rs_2, t_2', rc_2',
rs_2')$, such that $(t_1, rc_1, rs_1) = (t_2, rc_2, rs_2)$ and $t_2' \neq
null$, then $true$ is returned.

\begin{algorithm}[t]
\small{
	\floatname{algorithm}{Listing}
	\caption{Redis SOR-Set: \textit{lookup}}
 	\label{alg:lookup}
 	\begin{algorithmic}[1]
 	  \Function{Lookup}{$e$}
 	    \State $ids$ $\gets$ \texttt{smembers} \underline{ids}:$e$
 	    \ForAll{$gid_{1}$ \In $ids$}
 	      \If{\texttt{exists} \underline{element}:$gid_{1}$}
 	        \State ($t_{1}$, $rc_{1}$, $rs_{1}$, $t_{1}'$) $\gets$
 	        \Statex\hspace{\algorithmicindent}\hspace{\algorithmicindent}\hspace{\algorithmicindent}\hspace{\algorithmicindent} \texttt{hmget} \underline{element}:$gid_{1}$ \underline{add.t} \underline{add.rc} \underline{add.rs} \underline{rmv.t}
 	        \If{$t_{1}' = \Null$}
 	          \State $lookup \gets \True$
 	          \ForAll{$gid_{2}$ \In $ids$}
 	            \If{\texttt{exists} \underline{element}:$gid_{2}$}
 	              \State ($\_$, $t_{2}$, $rc_{2}$, $rs_{2}$, $t_{2}'$, $rc_{2}'$, $rs_{2}'$) $\gets$
 	              \Statex\hspace{\algorithmicindent}\hspace{\algorithmicindent}\hspace{\algorithmicindent}\hspace{\algorithmicindent}\hspace{\algorithmicindent}\hspace{\algorithmicindent}\hspace{\algorithmicindent} \texttt{hgetall} \underline{element}:$gid_{2}$
 	              \If{($t_{1}$, $rc_{1}$, $rs_{1}$) = ($t_{2}$, $rc_{2}$, $rs_{2}$) \Land
 	              \Statex\hspace{\algorithmicindent}\hspace{\algorithmicindent}\hspace{\algorithmicindent}\hspace{\algorithmicindent}\hspace{\algorithmicindent}\hspace{\algorithmicindent}\hspace{\algorithmicindent}\hspace{-3pt} $t_{2}' \neq \Null$}
 	                \State $lookup \gets \False$
 	                \State \Break
 	              \EndIf
 	            \EndIf
 	          \EndFor
 	          \If{$lookup = \True$}
 	            \State \Return \True
 	          \EndIf
 	        \EndIf
 	      \EndIf
 	    \EndFor
 	    \State \Return \False
 	  \EndFunction
	\end{algorithmic}
 }
\end{algorithm}

Procedures {\small\textsc{Add}}, {\small\textsc{Remove}}, and
{\small\textsc{Lookup}} execute atomically and in isolation with other Redis
commands on the store.

\textbf{Merge}. The code for last method is given in Listing~\ref{alg:merge}.
For lack of space we present only the main subroutines. On lines 2 and 3
we compute the minimum versions for both the local and the remote clusters
using the formula explained in Section~\ref{sec:sharding}. Based on the local
version $\tilde{T}_{x}$, {\small\textsc{GetUpdates}} fetches the updates from
remote cluster $rc_{y}$. This is done in 2 steps: first it gets the ids of
the updates using \underline{index}:$rc$:$rs$ and then it retrieves the actual
elements. {\small\textsc{AddUpdates}} distributes these elements to the stores
in the local cluster according to $hash_{x}$ function. Adding an update element
to the store is an operation similar to \textit{add}, except that the logical
clock is not incremented and the TTLs are the ones retrieved before. Setting the
logical clocks is done at the end in the {\small\textsc{UpdateTimestamps}}
subroutine according to the formula $T_{x}^{j} \coloneqq max(T_{x}^{j},
\tilde{T}_{y}), \forall j \in \{1,\ldots,|rc_{x}|\}$.

\begin{algorithm}[t]
\small{
	\floatname{algorithm}{Listing}
	\caption{Redis SOR-Set: \textit{merge}}
 	\label{alg:merge}
 	\begin{algorithmic}[1]
 	  \Procedure{Merge}{$rc_{x}$, $rc_{y}$}
 	    \State $\tilde{T}_{x}$ $\gets$ \Call{Version}{$rc_{x}$}
 	    \State $\tilde{T}_{y}$ $\gets$ \Call{Version}{$rc_{y}$}
 	    \State $updates$ $\gets$ \Call{GetUpdates}{$rc_{y}$, $\tilde{T}_{x}$}
 	    \State \Call{AddUpdates}{$rc_{x}$, $hash_{x}$, $updates$}
 	    \State \Call{UpdateTimestamps}{$rc_{x}$, $\tilde{T}_{y}$}
 	  \EndProcedure
 	\end{algorithmic}
}
\end{algorithm}
