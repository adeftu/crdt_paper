\section{Conclusions}
\label{sec:conclusions}

Achieving consistency in large-scale distributed systems is not an easy task. To
make things more difficult, designers need to also ensure high-throughput,
low-latencies accesses to the databases. However, building reliable distributed
systems demands trade-offs between consistency and availability as stated by the
CAP theorem~\cite{Gilbert:2002:BCF:564585.564601}. Eventual consistency is a
technique of compromise, widely adopted, but lacking a rigorous theoretical
foundation which makes current approaches ad-hoc and
error-prone~\cite{DeCandia:2007:DAH:1294261.1294281}.

The concept of CRDTs defines replicated data types that have mathematical
properties conferring them a form of eventual consistency, strong eventual
consistency. This model can be described from two equivalent perspectives:
a) state-based: object replicas apply updates locally and later exchange and
merge their states, and b) operation-based: update operations are distributed
among replicas over a reliable broadcast communication channel. Both approaches
guarantee convergence towards a common state without application-level conflict
resolutions, roll-backs, or consensus among
replicas~\cite{Terry:1995:MUC:224056.224070}. Because this model imposes strong
constraints on the type specification, it is not universal though.

In this paper we focused on designs for conflict-free replicated sets and gave
number of practical examples. We introduced two improvements to the original
OR-Set specification: an algorithm for efficient delta-based synchronization and
an extension for replica sharding. Lastly, we proposed a garbage collection
mechanism to support lifetime-limited elements and to alleviate the problem of
unbounded database growth. On the practical side, to the authors' knowledge,
this is the first implementation of a CRDT in the sense of the system model
described in this chapter. Proof-of-concept examples exists~\cite{ericmoritz,
dominictarr}, but they focus only on testing the specifications for CRDTs
locally and in-memory, without a real database store support.

Based on this type, more complex structures can be built. Maps can be
implemented as sets of registers and graphs can include two sets: one for
vertices and another for edges.
